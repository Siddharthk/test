In this thesis, we presented component based modeling in scene images. The decomposition
of texture image into the direct and the global components preserves sharpness of shadows and also models
specular reflection. This causes image to appear more photorealistic and single
point source effect is more prominent. This technique results in enhanced
photorealism which preserves sharp shadows and specular properties from
smoothening out. The separate model for luminance estimation provides us with
color values which are in close agreement with the color values of the original
image. The advantage of the technique is that photorealism can be achieved 
without accurate modeling of complex real-world physical interactions. We capture
lighting effects directly as they appear in reality. As they are captured in images, complex
interactions like self-shadowing, inter-reflections and sub-surface scattering can be reproduced automatically. It also does not
depend on the complexity of the scene and the surface properties of objects in the scene. 
It depends on the number of images captured or on the representation rather than the complexity of the scene. 
Results obtained on re-rendering the input images show a great
improvement over original PTM technique. Applying this technique over the inscribed text also helps us to extract
out text from degraded textured surfaces.

ICA decomposition when applied on natural scene images helps us to separate the foreground(text) from the complex/textured background.
It is an effective method to binarize text from colored scene text images containing reflective, shadowed and specular
background. By using a blind source separation technique followed by global thresholding, we are able to clearly separate
the text portion of the image from the background. It enables us to separate reflections, shadows and 
specularities from natural scene texts so that the global thresholding methods can be applied afterwards to binarize the 
text image. Experimental results on ICDAR dataset demonstrate the superiority of our method over other existing methods. 
Possible directions for improvement of the approach includes a patch-based SVM classification for thresholding as well
as integration of the results with a spatially aware optimization such as MRF. Working with text where the foreground and
background have same color is also of great interest.
 


% 
% One common disadvantage of image-based rendering methods is the amount of data storage required. As a result 
% of using a database of images as the representation rather than compact mathematical models storage sizes can be
% very large. Some techniques, such as the PTM method just described, avoid some of the data storage by 
% approximating the images with a simple but powerful model. Another common limitation of image-based 
% methods arises out of sampling issues. It can be difficult to capture high-frequency effects accurately without 
% extremely dense (and time consuming) sampling. Editing can also be problematic for these methods, depending 
% on the representation.
% Despite these limitations image-based rendering and relighting can be very useful for games because of their 
% significant advantages. The recent great leaps in capabilities of graphics hardware have made many more image 
% based techniques useful for a wide range of games and other applications.